\problem{Big Difference}
{"Big Difference!" Ivan said to his teacher, pointing at a list of numbers the teacher wrote on the board.

"That's right Ivan, there is a big difference here", the teacher said pointing to how the difference between every two elements in the list is fairly large. "In fact, I had carefully ensured that the difference between all elements was as large as possible, so that no elements next to each other was small." Specifically, the teacher made a list where the smallest (absolute) difference between every adjacent element was as large as possible.

Ivan wanted to figure out how to do this, but was not very fast at it. He couldn't figure out a way to solve it other than trying every different arrangement of the list, which was far too slow when he had a large list to deal with.}
{Input consists of a single integer $n$ where $1 \le n \le 10^8$.}
{Output an list of numbers of size $n$ with values $[1,2,3,...,n]$ where the smallest absolute difference between adjacent elements is maximized. More formally, create a permutation of an array $a$ with values $[1,n]$ with the maximum value of $min(|a_1-a_2|, |a_2-a_3|, ... |a_{n-1}-a_{n}|)$. If there are multiple answers, print any.}
{1 second}
{1024 mb}
{\IOsample{problems/big_difference/1}
\IOsample{problems/big_difference/2}}