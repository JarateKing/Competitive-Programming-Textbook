\documentclass[12pt]{article}

% set margins on page
\usepackage{geometry}
\geometry{a4paper,
		  lmargin=2.5cm,
		  rmargin=2.5cm,
		  tmargin=2.5cm,
		  bmargin=3.0cm}

% paragraph formatting
\setlength{\parskip}{0.75em}
\setlength{\parindent}{0em}

% remove margin on lists
\usepackage{enumitem}
\setlist{nolistsep}

% improve text rendering
\usepackage[T1]{fontenc}
\usepackage{lmodern}

% support links
\usepackage{hyperref}

% title page image support
\usepackage{pdfpages}

% additional support
\usepackage{minted}
\usepackage{tcolorbox}

\usepackage{xcolor}
\definecolor{CodeBGclr}{rgb}{0.935,0.935,0.935}

\newcommand{\inputcode}[2]{
\inputminted[
  fontsize=\small,
  tabsize=4
]{#1}{#2}
}

\newcommand{\inputtcolorbox}{
\begin{tcolorbox}[
  width=\linewidth,
  colframe=CodeBGclr,
  colback=CodeBGclr,
  boxsep=1mm,
  arc=3mm
]
}

\newcommand{\inputcpp}[1]{
\inputtcolorbox
\inputcode{cpp}{#1}
\end{tcolorbox}
}

\newcommand{\inputpython}[1]{
\inputtcolorbox
\inputcode{python}{#1}
\end{tcolorbox}
}

\newcommand{\inputjava}[1]{
\inputtcolorbox
\inputcode{java}{#1}
\end{tcolorbox}
}

\newcommand{\IOleftBox}[1]{
\begin{tcolorbox}[
  width=.49\linewidth,
  colframe=CodeBGclr,
  colback=CodeBGclr,
  boxsep=1mm,
  arc=3mm,
  nobeforeafter,
  sharp corners=east,
  equal height group=#1
]
}

\newcommand{\IOrightBox}[1]{
\begin{tcolorbox}[
  width=.49\linewidth,
  colframe=CodeBGclr,
  colback=CodeBGclr,
  boxsep=1mm,
  arc=3mm,
  nobeforeafter,
  sharp corners=west,
  equal height group=#1
]
}

\newcounter{IOsampleGroupCounter}

\newcommand{\IOsample}[1]{

\stepcounter{IOsampleGroupCounter}

\texttt{Input:} \hfill \hspace{4mm} \texttt{Output:} \hfill \hspace{1mm}

\vspace{-2mm}
\IOleftBox{\arabic{IOsampleGroupCounter}}
\inputcode{cpp}{#1.in}
\end{tcolorbox} \hfill
\IOrightBox{\arabic{IOsampleGroupCounter}}
\inputcode{cpp}{#1.out}
\end{tcolorbox}
} % code snippets
\newcommand{\problem}[7]{
\index{Problems!#1} \section*{#1}
\vspace{-5mm}
#2

\vspace{-5mm}
\subsection*{Input}
\vspace{-5mm}
#3

\vspace{-5mm}
\subsection*{Output}
\vspace{-5mm}
#4

\vspace{-5mm}
\subsection*{Constraints}
\vspace{-5mm}
Time Limit: #5 \hfill

\vspace{-3mm}
Memory Limit: #6 \hfill

\vspace{-5mm}
\subsection*{Samples}
\vspace{-5mm}
#7
} % problem description

\begin{document}

\section{Introduction}
\subsection{Recreational Programming}
\subsection{Programming Contests}
\subsection{Programming Sites}
\subsection{Purpose of This Book}
\subsection{Tips}
\subsection{Example Problems}

\hspace{0mm}

\section{Programming Environment}
\subsection{C++}
\subsubsection{Setup}
\subsubsection{Strengths and Weaknesses}
\subsection{Java}
\subsubsection{Setup}
\subsubsection{Strengths and Weaknesses}
\subsection{Python}
\subsubsection{Setup}
\subsubsection{Strengths and Weaknesses}
\subsection{Other}
\subsubsection{C}
\subsubsection{Pascal}
\subsubsection{Kotlin}
\subsection{Text Editors}

\hspace{0mm}

\section{Algorithm Analysis}
\subsection{Big-O Notation}
\subsection{Runtime Complexity}
\subsection{Space Complexity}
\subsection{Precision}
\subsubsection{Integer Types}
\subsubsection{Floating Point Types}

\hspace{0mm}

\section{Problem Analysis}
\subsection{IO Formats}
\subsubsection{Interactive IO}
\subsection{Problem Solving}
\subsection{Reading Comprehension}
\subsection{Debugging}

\hspace{0mm}

\section{Foundational Data Structures}
\subsection{Arrays}
\subsubsection{Dynamic Arrays}
\subsection{Sets}
\subsection{Maps}
\subsection{Sequential Structures}
\subsubsection{Stack}
\subsubsection{Queue}
\subsubsection{Deque}
\subsubsection{Priority Queue}

\hspace{0mm}

\section{Foundational Algorithms}
\subsection{Sorting}
\subsection{Searching}

\hspace{0mm}

\section{Iterative Problems}
\subsection{Brute-Force}
\subsection{Simulation}
\subsection{Two-Pointers}
\subsection{Sliding Window}
\subsection{Constructive}

\hspace{0mm}

\section{Greedy Problems}

\hspace{0mm}

\section{Dynamic Programming Problems}

\hspace{0mm}

\section{Advanced Data Structures}
\subsection{Range-based Structures}
\subsubsection{Fenwick Tree / BIT}
\subsubsection{Segment Tree}

\hspace{0mm}

\section{String Problems}
\subsection{String Concepts}
\subsubsection{Anagrams}
\subsubsection{Palindromes}
\subsubsection{Prefixes}
\subsubsection{Suffixes}
\subsection{Pattern Matching}
\subsubsection{Regex}
\subsubsection{Z-Algorithm}
\subsubsection{Knuth-Morris-Pratt}
\subsubsection{Boyer Moore}
\subsubsection{Aho Corasick}
\subsection{Subsequence Matching}
\subsection{Hashing}
\subsection{Distance}
\subsubsection{Diff}
\subsubsection{Levenshtein}

\hspace{0mm}

\section{Graph Problems}

\hspace{0mm}

\section{Number Theory Problems}

\hspace{0mm}

\section{Combinatorics Problems}

\hspace{0mm}

\section{Geometry Problems}

\hspace{0mm}

\section{Statistics Problems}

\hspace{0mm}

\section{Game Theory Problems}

\pagebreak

\problem{Hello World!}
{Print out Hello World!}
{There is no Input}
{Output is a single string, Hello World!}
{1 second}
{1024 mb}
{\IOsample{problems/hello_world/1}}

\hrulefill

The solution in various languages look like this:

\inputcpp{code/hello_world_solution.cpp}
\inputjava{code/hello_world_solution.java}
\inputpython{code/hello_world_solution.py}

\end{document}