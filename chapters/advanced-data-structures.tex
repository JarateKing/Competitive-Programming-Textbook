\section{Advanced Data Structures}
\index{Data Structures}

"Advanced" is possibly a misnomer. Some of these data structures aren't very complicated, in either their implementation or their usage. We call them advanced here because they relate to much more specific and particular applications. Things like arrays or sets are necessary to solve many basic problems -- while things like range-based structures may be necessary to efficiently implement an algorithm or solution, the problems they appear in are never as simple.

\subsection{Concepts}

\subsubsection{Persistent Data Structures}
\index{Persistent Data Structures}

Persistent data structures are ones that allow for not just querying an element, but also being able to query it at a specific time, keeping track of the modifications that have been done to it. For example, if you had a queue with 10 elements and pushed and popped several times, you wouldn't be able to query what the front of the queue is at its initial state with 10 elements. A persistent queue would not only allow you to query it at that state, or after any arbitrary amount of pushes and pops for that matter.

A trivial example of a persistent data structure is simply a list of copies of a complete object for every modification that happens to it. If you add an element to a persistent queue, you keep a copy of the old queue and create a new copy of the list with that element added. Likewise with popping the first element in the queue. You do this 100 times, you now have 100 more copies of the queue at different points in history than you started with. And if you want to query the fron of the queue at some point in history, you can easily refer to the proper state and use that.

Conceptually, this is a persistent data structure -- it managed to keep track of its history, and allows us to query this at any time. It should be noted that this is relatively slow, however -- copying the queue is $O(n)$ and is required each time a modification is made to the queue, which makes it too slow for most problems that may require a persistent data structure. Similarly, it takes a large amount of memory, and in problems with fairly large numbers of elements or operations this has a strong likelihood of using too much memory.

Ideally, persistent data structures involve representing the data in such a way that the history is implicit to the structure itself -- that copying the entire structure is unneeded, and it's preferable to always add to the structure in some way instead.

In our persistent queue's case, a better approach would be to recognize that we only ever can represent a queue as a subarray of all the elements added in order -- if we have an array that contains all the elements that will be added to our queue, we can keep track of the index of the element we consider at the front of our queue, and the total size of our queue from that starting position (alternatively, the end position of our queue, or one more than the index of our last element in the queue). Pushing is a simple matter of incrementing our size by 1, while popping involves incrementing our start position by 1 and decrementing our size by 1 as well. For every push or pop we perform on our persistent queue, we can keep track of the state of the queue as a pair of integers for the start position and the size of the queue:

\inputcpp{code/data_structures/persistent_queue.cpp}

\subsubsection{Lazy Evaluation}
\index{Lazy Evaluation}

In some cases, we are far more likely to update a data structure than we are to query it. In other cases, we may have many updates in one go before we make any queries. Perhaps an update involves modifying a huge range of elements in our data structure, when we only perform queries on a small handful of them, and most of the updates are ignored and unused.

For data structures where updating is $O(1)$ it's most likely fine to update as normal. If our updates happen in $O(log n)$ or $O(n)$ or some other non-constant time complexity, we may begin to be concerned with updates. Specifically, we would like to batch our updates together until we actually perform a query that's affected by that update.

Consider a data structure where...

We can make these updates lazy as...

\subsection{Range-based Structures}
\index{Range-Based Data Structures} \index{Subarray Sums}

Under Construction

\subsubsection{Prefix Sum Arrays}
\index{Prefix Sum Arrays}

Prefix sums are not a particularly complicated data structure. However, they are useful as the most basic form of range-based data structures. As the name suggests, they deal with sums of integers.

If we are given an array of integers, we can generate a prefix sum array by making each element be the cumulative sum of all elements before it in the original array. For example, given an array $arr = \{1,2,3,5,10,3,2,5\}$, our prefix sum is $ps = \{1,3,6,11,21,24,26,31\}$ because each $ps_i$ is equal to $arr_0 + ... + arr_i$.

Generating the prefix sum is a simple process...

Normally, the prefix sum gives us the sum of the array from $[0,i)$. We can obtain the sum of an arbitrary subarray $[l,r)$ by obtaining $[0,r)$ and subtracting $[0,l)$. 

While for some problems prefix sums are sufficient (construction in $O(n)$ and queries in $O(1)$), they aren't able to be updated efficiently (you would have to recalculate all sums after the element you're updating, which is $O(n)$).

\subsubsection{Sparse Table}
\index{Sparse Tables}

Under Construction

\subsubsection{Fenwick Tree / Binary Indexed Tree}
\index{Fenwick Trees} \index{BITs}

Fenwick Trees, also commonly known as Binary Indexed Trees (which is also commonly abbreviated to BITs) is a structure that allows calculates the prefix sum that can be queried in $O(log n)$ and allows for updates in $O(log n)$ as well, compared to the prefix sum array that allows $O(1)$ queries but $O(n)$ updates.

\subsubsection{Segment Tree}
\index{Segment Trees}

Segment Trees are conceptually similar to Fenwick Trees, in that they support range queries in $O(log n)$ and updates in $O(log n)$ as well. Segment Trees have the notable benefit of supporting more functions than just summation, though -- sums, products, maximum, minimum, GCD\index{Greatest Common Divisor}, LCM\index{Least Common Multiple}, or xor\index{XOR} of an arbitrary range are some examples that appear with some frequency.

The cost of this is some slight overhead -- there is more code to them than Fenwick Trees, and they require some additional memory as well, but generally most problems that can be solved with Fenwick Trees can also be solved with Segment Trees.

\subsubsection{Wavelet Tree}
\index{Wavelet Trees}

Wavelet Trees are range based structures that lend themselves to several different operations -- namely, counting the number of occurrences of a value in a range, the number of occurrences of between two values within a range, and getting the $k$th smallest value in a range.

Construction of the Wavelet Tree can be done as...

Counting the occurrences of a value in a range can be done as...

The number of occurrences of any number between two values can be done as...

The $k$th smallest value in a range can be implemented as...

\subsection{Ropes}
\index{Ropes}

Ropes are a data structure that can be thought of as an alternative to strings, with some operations that have better time complexity and others that are worse.

Inserting or deleting a character at the end of a string is generally a $O(1)$ operation (amortized), but insertion or deletion at an arbitrary point in the string is $O(n)$. For ropes, the time complexity of insertion or deletion is $O(log n)$ at any arbitrary point, but is also $(log n)$ when inserting or deleting at the end of the string. One of the general costs with this is that querying or modifying a character at an arbitrary index is $O(1)$ in strings, but is an $O(log n)$ operation in ropes.

\hrulefill

\problem{Some Sums}
{Sometimes we want to sum a range of an array. Some call this a "range sum". Sometimes, we find some amount of range sums, and sum them together. Some would consider this a sum of some range sums problem.}
{Input starts with two numbers, $n$ and $m$ where $1 \le n \le 10^5$ and $1 \le m \le 10^5$. For every $n$ there is a value $a_i$ where $-5000 \le a_i \le 5000$. For every $m$ there is a two integers $x_i,y_i$ representing the range $[x_i,y_i)$ where $0 \le x_i < y_i \le n$.}
{Output the total sum of all range sums. In other words, calculate and print $\sum_{i=0}^{m-1} \sum_{j=x_i}^{y_i - 1} a_j$.}
{1 second}
{1024 mb}
{\IOsample{problems/some_sums/1}
\IOsample{problems/some_sums/2}}

\hrulefill
