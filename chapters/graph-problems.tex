\section{Graph Problems}
\subsection{Concepts}
\subsubsection{Directed and Undirected}
\subsubsection{Cyclic and Acyclic}
\subsubsection{Weighted}
\subsubsection{Transitive Closure}
\subsubsection{Grids}
\subsection{Pathfinding}

One of the most common tasks with graphs is to find a path between two nodes, or the distance between them. As we will see, traversing the graph in this manner is a common subtask in several other, more complicated graph algorithms as well.

It's important to know that, in general, there isn't any single pathfinding algorithm that you should always use. Each have their own strengths and weaknesses that are only relevant in certain circumstances, or come with specific restrictions that make them only worthwhile on specific types of graphs.

\subsubsection{Depth First Search}
\subsubsection{Breadth First Search}
\subsubsection{Djikstra's Algorithm}
\subsubsection{A* Algorithm}
\subsubsection{Bellman-Ford Algorithm}
\subsubsection{Floyd-Warshall Algorithm}
\subsection{Union-Find}

Union-Find, which also goes by the name of \textit{disjoint-set}, is a data structure that groups connected nodes together into different subsets through operations called \textbf{union} (which connects two nodes together, akin to creating an edge) and \textbf{find} (which determines which subset a node belongs to).

\subsection{Minimum Spanning Trees}
\subsubsection{Prim's Algorithm}
\subsubsection{Kruskal's Algorithm}
\subsection{Strongly Connected Components}
\subsubsection{Kosaraju's Algorithm}
\subsubsection{Tarjan's Algorithm}
\subsection{Maximum Flow}
\subsection{Bipartite Graphs}
\subsubsection{Determining Bipartite Graphs}
\subsubsection{Maximum Bipartite Matching}
\subsection{Eulerian Paths}
\subsubsection{BEST Algorithm}
\subsubsection{Hierholzer's Algorithm}