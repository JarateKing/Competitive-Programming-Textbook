\section*{Preface}

\subsection*{For a Beginner Programmer}

For those who are relatively new to programming, this book can serve as an excellent resource. There are plenty of exercises that a new programmer can practice on, and there are many different concepts they will be exposed to throughout this book.

That said, this would not do well as your only resource. The book expects that you either know one of C++, Java, or Python, or that you are able to learn what you need from other books or websites when you need to. While this is meant to be usable by someone relatively new to programming, it isn't meant to cover everything needed to learn programming from scratch. This serves well as a "second step" for new programmers by providing tasks and concepts beyond an introduction to programming (that many beginner programmers are unable to find resources for), without being unaccessible to someone with relatively little experience.

\subsection*{For a Beginner Competitive Programmer}

If you're fairly comfortable with the languages you're programming in, but want to learn how to program competitively, this book is ideal. We focus on many simple applications of common data structures and algorithms, and offer many more creative applications that are intended to further develop a budding competitive programmer's skills.

That isn't to say that people interested in things other than competitive programming won't find value in this book. Technical interviews will commonly have similar tasks to what is presented in this book and involve the same methods and techniques to solve. Many data structures and algorithms are presented that are not as easy to find as normal within other books or resources, but have practical use that can be worthwhile to know. Often, concepts taught and utilized for competitive programming is applicable in other domains as well. You don't necessarily need to participate in competitive programming (though this book certainly recommends it) in order to benefit from learning about it.

\subsection*{For an Experienced Competitive Programmer}

A competitive programmer who has their share of experience will be able to skip large portions of this book, since much of it is foundational knowledge that they would already be familiar with. That said, there are many less well-known algorithms and data structures that should be of interest, and many problems that can be good to solve.

\subsection*{For a Teacher}

If you teach a course on data structures or algorithms, you will likely see a benefit to this book. There's a focus on the applications of data structures and algorithms, rather than the approach many textbooks take where they focus on the design and implementation details of data structures and algorithms. Competitive programming tasks are an excellent form of learning and developing intuition about these data structures and algorithms, because they:
\begin{itemize}
\item are very well-defined tasks, backed with testing
\item range from very standard to incredibly creative applications of data structures and algorithms
\item are usually not implementation-heavy and won't bog down students with unnecessary details, they're generally meant to be quick to implement, as long as you're quick to figure out a valid solution.
\end{itemize}

\subsection*{For a Coach}

For those that coach students and competitors in competitions, this book can provide a good list of topics to cover as well as various example problems to cover. The focus is, first and foremost, to provide a resource for potential competitors to become familiar with competitive programming and get up to speed on many of the necessary (or additional) concepts. For coaches, you can do anything from getting students to read this book directly, to covering the exact topics and problems here, to using this as a guideline and suggestions as you prepare your own training material.