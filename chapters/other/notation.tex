\section*{Notation}
\index{Notation}

A brief description of some of the mathematical notation used in this book might be useful, because not everyone reading this may be overly familiar with some of them:
\begin{itemize}
\item when describing a list or set, we will usually denote it as $\{1,2,3,4,5\}$. Multiple sets or lists have the notation $\{1,2\},\{3,4\},\{5,6\}$.
\item inclusive ranges are given in the form of $[x,y]$ that represents every value beginning with $x$ (including $x$) and ending with $y$ (including $y$). For example, the range $[1,5]$ includes the values $\{1,2,3,4,5\}$.
\item ranges may also exclude a value, by using $($ or $)$ instead of $[$ and $]$. The one most often encountered would be $[x,y)$, where we have a range from $x$ (including $x$) to $y$ (not including $y$). As a result, $[1,5)$ is the values $\{1,2,3,4\}$.
\item modular arithmetic follows code-like conventions, where we use $\%m$ to denote numbers $mod$ $m$. For example, adding $2$ and $3$ under modulo $7$ would look like $(2 + 3) \% 7 = 5$ rather than the traditional $2+3 \equiv 5 \mod{7}$.
\end{itemize}

There is some specific notation we will use in this book as well, that we will cover here.
\begin{itemize}
\item individual elements of an array are addressed, 0-indexed, as $arr[i]$ where $arr$ is the array and $i$ is the index, or with constants such as $arr[0]$.
\item subarrays use the notion $arr[i:j]$ for elements in the range $[i,j)$, similar to how Python handles subarrays.
\item strings are surrounded by double-quotes like \mintinline{cpp}{"string"}.
\item individual characters are surrounded by single-quotes like \mintinline{cpp}{'c'}. Characters that use escape codes such as newlines are given as \mintinline{cpp}{'\n'}.
\end{itemize}

\section*{Terminology}
\index{Terminology}

There's also some terms used specifically for competitive programming. Some of these terms will be found in this book, while others will only be found in contest settings or in forums dedicated to competitive programming.

\begin{itemize}
\item \textbf{AC} is a shortened version of "accepted", used as a verdict for a problem submission when it is fully correct and passes all testcases within the required time and memory constraints.
\item \textbf{WA} is an abbreviation of "wrong answer", used as a verdict for a problem submission, when output is wrong for at least one testcase.
\item \textbf{TLE} is an abbreviation of "time limit exceeded", used as a verdict for a problem submission, when the program takes too long to solve some testcase.
\item \textbf{MLE} is an abbreviation of "memory limit exceeded", used as a verdict for a problem submission, when the program takes too much memory to solve some testcase.
\item \textbf{RTE} is an abbreviation for "runtime error", used as a verdict for a problem submission, when the program encountered a fatal error and prematurely terminated on some testcase. Usually, exiting the program with an error code is what causes an RTE verdict (even if the output is correct and the submission would be AC otherwise!).
\item \textbf{CE} is an abbreviation for "compile error", used as a verdict for a problem submission when the sourcecode submitted does not compile.
\item \textbf{RLE} is an abbreviation for "recursion limit exceeded", used as a verdict for a problem submission, when the program surpasses the language or system's maximum recursion depth. This verdict doesn't happen on every platform, since the RTE and MLE verdicts will already cover this.
\item \textbf{OJ} is an abbreviation for "online judge", a platform for competitive programming problems.
\item \textbf{orz} is an emoticon commonly used in competitive programming circles, to signify respect for someone (the emoticon itself depicts someone bowing down on hands and knees, with the \mintinline{cpp}{o} being the head, the \mintinline{cpp}{r} being the torso and arms, and \mintinline{cpp}{z} being the rest of the body and legs).
\end{itemize}

Other terms and abbreviations will be introduced as the topic is explored.