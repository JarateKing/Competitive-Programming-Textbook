\section{Templates}
\index{Templates}

Many competitive programmers, rather than remember the details for complicated data structures and algorithms, will save a copy of the necessary code for later reuse.

\subsection{Snippets}
\index{Snippets}

Code that you intend to use for templates can differ widely from other code you write -- it's often far more generalized than what you normally write in competitive programming, which makes it like general software development in that sense. But it's also usually far more compressed with little concern about architectural decisions, much more like competitive programming than general development.

\subsection{Team Notebooks}
\index{Team Notebooks} \index{ICPC}

Currently in ICPC competitions, and many smaller-scale competitions, you are allowed a 25-page single-sided notebook containing code snippets. This section contains some recommendations for creating this.

If you look for different ICPC notebooks, you will find many of them are written in latex. While alternatives (like word documents) are possible, latex is generally recommended because there are many options and packages that can include code snippets and fine-tune formatting easily. It tends to be easier to maintain and modify, and generally produces more lean or compressed final documents as well (with the proper settings) with far more customization options available to you than the majority of alternatives.

Some good examples of notebooks would be:

\begin{itemize}
\item \url{https://github.com/kth-competitive-programming/kactl/blob/master/kactl.pdf}
\item \url{https://github.com/SuprDewd/CompetitiveProgramming/blob/master/comprog.pdf}
\end{itemize}
