\section{Dynamic Programming Problems}

Dynamic programming is...

Consider the problem of calculating the Fibonacci sequence. The Fibonacci sequence is the sequence $0,1,1,2,3,5,8,13,21,34,55,89,...$ where each value (other than the initial $0,1$) is the sum of the previous two (so $89 = 55 + 34$, $55 = 34 + 21$, $34 = 21 + 13$, and so on). This is a pretty intuitive example of recursion, like in:

\inputcpp{code/dp/fib_recursive.cpp}

Unfortunately, this results in a lot of repeated work. A call to \mintinline{cpp}{fibonacci(10)} will call \mintinline{cpp}{fibonacci(9)} and \mintinline{cpp}{fibonacci(8)}, where \mintinline{cpp}{fibonacci(9)} will also call \mintinline{cpp}{fibonacci(8)}, \mintinline{cpp}{fibonacci(7)} ends up getting called 3 times, \mintinline{cpp}{fibonacci(6)} gets called 5 times, and so on until \mintinline{cpp}{fibonacci(1)} ultimately gets called 56 times. There is a huge amount of work being done with these recursive calls \textit{that we already obtained the result from earlier}.

Dynamic programming fixes this by storing the results of these recursive subcalls in a process called memoization.

\inputcpp{code/dp/fib_dp_td.cpp}

That is called "top down" dynamic programming because we still make recursive calls, but if we've already gotten the result from one of our recursive calls we'll store it for any times it appears later.

The alternative is "bottom up" dynamic programming where we start with the smallest cases and gradually work our way up:

\inputcpp{code/dp/fib_dp_bu.cpp}

Here, we don't make any recursive calls at all -- a call of \mintinline{cpp}{fibonacci(10)} will not result in recalculating earlier values in the sequence repeatedly.

You may note that we can store this permanently, by precalculating entirely:

\inputcpp{code/dp/fib_precalc.cpp}
