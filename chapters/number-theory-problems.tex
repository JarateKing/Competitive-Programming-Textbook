\section{Number Theory Problems}

\subsection{Binary Exponentiation}

\subsection{Primes}

\subsubsection{Prime Sieve}

\subsubsection{Primality Test}

\subsubsection{Prime Factoring}

\subsection{Greatest Common Divisor}

\subsubsection{Basic GCD}

\subsubsection{Extended GCD}

\subsubsection{Least Common Multiple}

The least common multiple between two numbers can be calculated quickly: $a * b * gcd(a, b)$.

\subsection{Modular Arithmetic}

\subsubsection{Addition}

Addition can be fairly simple -- we add our numbers together and then perform the modulo operation on it. For example:

\inputcpp{code/number_theory/modulo_add.cpp}

The modulo operation is relatively expensive for a basic arithmetic operation, however. If we can guarantee that $a$ and $b$ are both less than $m$, we can avoid doing any modulo operations with:

\inputcpp{code/number_theory/modulo_add_fast.cpp}

\subsubsection{Subtraction}

Subtraction is similar to addition, but we need to guarantee that we don't go negative. We can fix this by simply adding $m$ before performing our modulo operation:

\inputcpp{code/number_theory/modulo_sub.cpp}

Likewise with addition, we can also optimize our subtraction function to not perform any modulo operation:

\inputcpp{code/number_theory/modulo_sub_fast.cpp}

\subsubsection{Multiplication}

Multiplication can (mostly) be performed similar to the non-optimized method of addition and subtraction -- by simply doing the multiplication operation and then performing the modulo operation on that. However, we may be concerned with overflows.

Specifically, consider $m = 10^9 + 7$. The largest numbers we will multiply will be when both $a$ and $b$ are equal to $10^9 + 6$. $a * b$ is somewhat above $10^{18}$ which is outside of the range of what a 32-bit integer can hold. In a language like C++ or Java, we will have to ensure that we use 64-bit integers when performing our multiplication:

\inputcpp{code/number_theory/modulo_mul.cpp}

\subsubsection{Exponentiation}

\subsubsection{Division (Prime Modulo)}

\subsubsection{Modular Inverse}

\subsubsection{Division (Any Modulo)}

\subsubsection{Discrete Log}

\subsubsection{Square Root}
