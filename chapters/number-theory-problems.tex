\section{Number Theory Problems}

\subsection{Greatest Common Divisor}

\subsubsection{Basic GCD}

\subsubsection{Extended GCD}

\subsubsection{Least Common Multiple}

\subsection{Primes}

\subsubsection{Prime Sieve}

\subsubsection{Primality Test}

\subsubsection{Prime Factoring}

\subsection{Modular Arithmetic}

\subsubsection{Addition}

Addition can be fairly simple -- we add our numbers together and then perform the modulo operation on it. For example:

\inputcpp{code/number_theory/modulo_add.cpp}

The modulo operation is relatively expensive for a basic arithmetic operation, however. If we can guarantee that $a$ and $b$ are both less than $m$, we can avoid doing any modulo operations with:

\inputcpp{code/number_theory/modulo_add_fast.cpp}

\subsubsection{Subtraction}

Subtraction is similar to addition, but we need to guarantee that we don't go negative. We can fix this by simply adding $m$ before performing our modulo operation:

\inputcpp{code/number_theory/modulo_sub.cpp}

Likewise with addition, we can also optimize our subtraction function to not perform any modulo operation:

\inputcpp{code/number_theory/modulo_sub_fast.cpp}

\subsubsection{Multiplication}

\subsubsection{Exponentiation}

\subsubsection{Inversion}

\subsubsection{Division}

