\section{Introduction}
\subsection{Recreational Programming}

Programming can be very productive -- making something with practical value, that people can use in their life to simplify tasks. Programming can also be done for the sake of programming itself, and only because it's fun. The latter is called "recreational programming", where the act of programming is done for fun rather than to create something useful.

There is a wide variety of forms of programming that people do recreationally. Many of them involve playing with how the sourcecode of a program appears, such as:
\begin{enum}
\item Codegolf -- solving problems in as few characters as possible
\item Polyglots -- writing programs that can successfully run in multiple different languages (either by giving the same output, or intentionally giving different outputs depending on the language)
\item Quines -- programs that output their own sourcecode
\item Competitive Programming - solving well-defined problems under specific constraints (runtime, memory) often in a tournament setting with an emphasis on solving quickly
\end{enum}

This book focuses on competitive programming.

\subsection{Competitive Programming}

The concept behind competitive programming is simple: you are given a problem with specific inputs and required outputs, and you have to submit code that solves it under well-defined constraints. These constraints are almost always in the form of a runtime limit and a memory limit. These are tested by an external "judge" that verifies your program works on hidden testcases (so that you can't hardcode the solution for every testcase) and that it works under the constraints (the runtime would depend on the computer that the judge is running on, but it should be relatively consistent for everyone submitting so no one gets an advantage or disadvantage).



\subsection{Programming Contests}


\subsection{Programming Sites}
\subsection{Purpose of This Book}
\subsection{Tips}
\subsection{Example Problems}