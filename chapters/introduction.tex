\section{Introduction}
\subsection{Recreational Programming}

Programming can be very productive -- making something with practical value, that people can use in their life to simplify tasks. Programming can also be done for the sake of programming itself, and only because it's fun. The latter is called "recreational programming", where the act of programming is done for fun rather than to create something useful.

There is a wide variety of forms of programming that people do recreationally. Many of them involve playing with how the sourcecode of a program appears, such as:
\begin{enum}
\item Codegolf -- solving problems in as few characters as possible
\item Polyglots -- writing programs that can successfully run in multiple different languages (either by giving the same output, or intentionally giving different outputs depending on the language)
\item Quines -- programs that output their own sourcecode
\item Competitive Programming - solving well-defined problems under specific constraints (runtime, memory) often in a tournament setting with an emphasis on solving quickly
\end{enum}

This book focuses on competitive programming.

\subsection{Competitive Programming}

The concept behind competitive programming is simple: you are given a problem with specific inputs and required outputs, and you have to submit code that solves it under well-defined constraints. These constraints are almost always in the form of a runtime limit and a memory limit. These are tested by an external "judge" that verifies your program works on hidden testcases (so that you can't hardcode the solution for every testcase) and that it works under the constraints (the runtime would depend on the computer that the judge is running on, but it should be relatively consistent for everyone submitting so no one gets an advantage or disadvantage).

There are several skills tested in competitive programming. Often, they rely on knowledge and creative applications of algorithms. Other times the solution does not need any particularly advanced techniques or prior knowledge, but tests your problem solving skills in an interesting way. Many times, it's easy to come up with a valid solution that, given enough time and memory, will get you the proper answer -- but the time constraints mean you need to develop a much more efficient approach to the problem. Some have a strong basis in math, while others are solely determined by computer science, and many combine mathematical and computational aspects together.

All of the above problem types take time to improve at. Sometimes they will be problems you are familiar with in your education and you can comfortably solve them, sometimes they will be advanced applications of techniques you've learned that are significantly more difficult than what you've likely encountered before, and sometimes they will involve techniques most people would never hear of outside of competitive programming.

\subsection{Value of Competitive Programming}

Competitive programming is very helpful to a programmer. For students, many homework tasks will be of a similar style to competitive programming tasks, and being skilled with competitive programming will give you a huge advantage in this sort of work. For those interested in academia, competitive programming is a great way to apply various algorithmic or computational techniques and to explore a wide variety of interesting problems to research. And for those looking to get hired by a company, not only does competitive programming experience look good on a resume, but many technical interviews involving algorithmic problems are essentially competitive programming problems. Someone with experience in competitive programming will have significant experience in all of the above that can be difficult to practice otherwise.

Not to mention that practicing competitive programming is a great way to practice programming in general. When learning a new language, it is useful to solve many smaller problems with it to get more comfortable. Fortunately, competitive programming practice involves many small problems to solve. When learning a new technique or algorithm, it's useful to find many problems that require it of varying difficulty. Often, competitive programming practice will list out problems in terms of topic and difficulty for this reason.

\hrulefill

It's important to preface this book by saying that competitive programming won't help all your skills programming. Competitive programming tasks are a niche within programming, and as a result have special considerations. The biggest being that code maintainability doesn't matter at all for competitive programming, and as a consequence clean code practices can be pretty much entirely ignored (and often are, for the sake of typing faster) as long as you're still able to debug your solution in case of an error.

Usually you don't care about things like memory management much either -- this is both a blessing and a cause for concern in a language like C++. It is generally much easier to learn and use C++ if you never concern yourself with manual memory management, as long as the solution runs through the testcases without passing the memory limit. But it is also missing out on a major part of the language that you will need if you work on larger projects.

Don't let that dissuade you though. These sorts of things develop through practicing or working with larger projects, that you absolutely can do alongside competitive programming.

\subsection{Programming Contests}
\subsection{Programming Sites}
\subsection{Purpose of This Book}
\subsection{Tips}
\subsection{Example Problems}