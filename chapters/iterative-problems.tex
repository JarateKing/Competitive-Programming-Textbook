\section{Iterative Problems}
\subsection{Brute-Force}

Brute force problems are best described as problems where the intended (or at least, a valid) solution involves trying every possibility. These are usually marked by very small problem bounds, because of how computationally complex this process can be.

What exactly brute-force looks like depends on the problem itself. In some cases, it's "try all pairs" which is $O(n^2)$. It can be "try all triplets" at $O(n^3)$. Sometimes it's all combinations, in $O(2^n)$. In some cases, it's try all permutations which is $O(n!)$.

There is a reason we distinguish brute-force \textit{problems} from brute-force \textit{solutions}. For most problems, it's possible to find a brute-force solution that is guaranteed to give you the right answer. Usually it's pretty easy to come up with a technically valid brute-force solution, in fact. Most problems can't be solved with this approach because it tends to be extremely slow. A brute-force problem is a problem where the brute-force solution is valid and can work.

\subsection{Simulation}

Simulation problems are problems where the problem description describes some step-by-step algorithm (and the algorithm may be a single step repeated some number of times), and the intended solution is to simply perform that algorithm.

\hrulefill

\problem{Very Happy}
{If you take a positive integer and sum the square of all its digits, you end up with an interesting transformation. For example, applying this to $123$ gives us $1^2 + 2^2 + 3^2 = 14$.

We can repeat this process multiple times, such as taking $14$ and getting $1^2 + 4^2 = 17$, $17$ becoming $1^2 + 7^2 = 50$, $50$ becoming $25$, $25$ becoming $29$, becoming $85$, then $89$, then $145$, then $42$, $20$, then $4$. This will continue to loop forever, since it turns out that $4$ enters a cycle $4,16,37,58,89,145,42,20,4$.

Not all numbers will enter into a long cycle when you do this -- some numbers will turn into $1$ when you repeat this process enough times (and applying this process on $1$ gives us $1^2 = 1$). These are called happy numbers.

The task here is to determine whether a number is happy or not. And if it is happy, figure out how many iterations it takes to do so.}
{Input consists of an integer $1 \le t \le 10^5$. Then follows $t$ integers $1 \le n \le 10^9$.}
{For every integer $n$, output either the number of iterations it takes to become $1$, or "impossible" if it never does.}
{1 second}
{1024 mb}
{\IOsample{problems/very_happy/1}}

\hrulefill

\subsection{Two-Pointers}

Two pointers problems, generally speaking, are problems that can be solved by keeping track of two locations on a sorted array. One pointer starts at the start of the array (the smallest element) and the other starts at the end of the array (the largest element). You are asked to find a pair of elements for which a certain property holds.

A classic example is "given an array of integers, determine whether a pair of elements sums up to 100".

It is simple to solve this problem using a brute-force approach in $O(n^2)$:

It can be improved further by using binary search instead our nested for loop, which turns it $O(n log n)$ instead:

But with a two-pointers approach we can perform this in $O(n)$:

\subsection{Sliding Window}
\subsection{Constructive}

Sometimes you will find problems along the lines of "here is a list of restrictions, construct an array that meets those restrictions" or "create the optimum answer under a list of restrictions". Usually, they also have some variant on "if there are multiple solutions, print any of them."

Generally, constructive problems require creative thinking to figure out a valid pattern that will satisfy the requirements. Usually, the implementation of a valid program itself is not very complicated, and it's being able to determine a valid pattern that is the challenging and interesting part of the problem.

Constructive problems are not necessarily iterative, but it's reasonable to introduce constructive problems here because many of them are.

We'll look at an example of a constructive problem:

\hrulefill

\problem{Big Difference}
{"Big Difference!" Ivan said to his teacher, pointing at a list of numbers the teacher wrote on the board.

"That's right Ivan, there is a big difference here", the teacher said pointing to how the difference between every two elements in the list is fairly large. "In fact, I had carefully ensured that the difference between all elements was as large as possible, so that no elements next to each other was small." Specifically, the teacher made a list where the smallest (absolute) difference between every adjacent element was as large as possible.

Ivan wanted to figure out how to do this, but was not very fast at it. He couldn't figure out a way to solve it other than trying every different arrangement of the list, which was far too slow when he had a large list to deal with.}
{Input consists of a single integer $n$ where $1 \le n \le 10^8$.}
{Output an list of numbers of size $n$ with values $[1,2,3,...,n]$ where the smallest absolute difference between adjacent elements is maximized. More formally, create a permutation of an array $a$ with values $[1,n]$ with the maximum value of $min(|a_1-a_2|, |a_2-a_3|, ... |a_{n-1}-a_{n}|)$. If there are multiple answers, print any.}
{1 second}
{1024 mb}
{\IOsample{problems/big_difference/1}
\IOsample{problems/big_difference/2}}

\hrulefill

There's a few things about this problem that stands out. The restriction is fairly basic -- we just have to make sure the minimum difference between adjacent elements is as large as possible. Our output is just a list of numbers between $1$ and $n$ that's ordered in a way that meets this restriction.

There can be multiple variants on this problem. It could also provide an input $k$ and ask that no two adjacent elements have an absolute difference less than $k$. It could ask for the lexicographically smallest answer if there are multiple (if $2 1 3$ and $3 1 2$ as both valid answers, the lexicographically smallest answer would be $2 1 3$ since $2$ is less than $3$).

It's worth mentioning that it's rare for the sample inputs/outputs to actually describe each possible different valid answer for a given input. It's also unlikely that the test cases will show particularly large test cases. The reason for both of these is that a valid pattern may become obvious with this information. The sample cases are usually carefully chosen so that it's clear what the IO expects and can clarify what the problem expects, but doesn't give you any strong leads into what a solution might look like.
