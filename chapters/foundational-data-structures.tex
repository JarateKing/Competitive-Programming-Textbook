\section{Foundational Data Structures}
\subsection{Arrays}

Arrays are the most basic data structure in most programming languages. It is very simple in its functionality: instead of storing a single value like a regular variable does, it stores some amount of values.

\subsubsection{Dynamic Arrays}

Arrays don't have to be a fixed size -- dynamic arrays allow you to dynamically resize the array when you need more space to put new elements.

In C++ we have the \mintinline{cpp}{std::vector} class that works as our dynamic array. In Java, it's \mintinline{java}{ArrayList}. And in Python, lists are already dynamic arrays.

\subsection{Sets}

In mathematics, a set is a collection of distinct elements. Pratically speaking, this differs from a traditional array in that every element is unique (in practice, trying to insert a duplicate element will not insert it) and that there is no guarantee of any specific order (in practice, the order depends on implementation).

In C++, the basic type for a set is \mintinline{cpp}{std::set}. Internally, this is implemented as a self-balancing tree, so that insertions, deletions, or queries all take $O(log n)$ time complexity.

C++ also offers the \mintinline{cpp}{std::unordered_set} that is based on a hash table. This offers $O(1)$ average case insertion, deletion, and queries, but is generally much less performant for iterating through the entire \mintinline{cpp}{std::unordered_set} than using an \mintinline{cpp}{std::set}.

\subsection{Maps}
\subsection{Sequential Structures}
\subsubsection{Stack}
\subsubsection{Queue}
\subsubsection{Deque}
\subsubsection{Priority Queue}