\section{String Problems}
\subsection{String Concepts}

\subsubsection{Anagrams}

Anagrams are the string equivalent to a permutation of an array. Two strings are anagrams of eachother if they contain the same elements, but their ordering isn't necessarily the same. For example, "abc" and "bac" are anagrams. Similarly, all anagrams of "abc" are \{"abc","acb","bac","bca","cab","cba"\}.

\subsubsection{Lexicographic Order}

Lexicographic order of strings is one where strings are sorted based on the alphabetical ordering of their first characters. When strings have the same first character, the alphabetical order of their second characters are considered, and so on. A similar tie-breaker has to appear with shorter strings, where the end of the string is considered first alphabetically.

For example, \{"abc", "abd", "ada", "ae", "aeiou", "b"\} is a lexicographically ordered array of strings, following the rules above.

\subsubsection{Palindromes}

Palindromes are strings that are the same when they're reversed. For example, "abccba" or "abcba" are both palindromes, whereas "abcde" isn't.

It can be pretty easy to check if a string is a palindrome or not. We can simply check whether the first half of characters are equal to the second half of characters.

\inputcpp{code/string/is_palindrome.cpp}

\subsubsection{Prefixes}

Prefixes are substrings that contain the start of a string. For example, the prefixes of "abcde" are \{"a","ab","abc","abcd","abcde"\}.

\subsubsection{Suffixes}

Opposite to prefixes, suffixes are substrings that contain the end of a string. For example, the suffixes of "abcde" are \{"e","de","cde","bcde","abcde"\}.

\subsection{Pattern Matching}
\subsubsection{Regex}
\subsubsection{Z-Algorithm}
\subsubsection{Knuth-Morris-Pratt}
\subsubsection{Boyer Moore}
\subsubsection{Aho Corasick}
\subsection{Subsequence Matching}
\subsection{Hashing}
\subsection{Distance}
\subsubsection{Diff}
\subsubsection{Levenshtein}