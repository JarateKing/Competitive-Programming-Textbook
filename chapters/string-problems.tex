\section{String Problems}
\subsection{String Concepts}

\subsubsection{Anagrams}

Anagrams are the string equivalent to a permutation of an array. Two strings are anagrams of eachother if they contain the same elements, but their ordering isn't necessarily the same. For example, "abc" and "bac" are anagrams. Similarly, all anagrams of "abc" are \{"abc","acb","bac","bca","cab","cba"\}.

We can check whether two strings are anagrams by comparing the count of each character they have in their string:

\inputcpp{code/string/is_anagram_2.cpp}

That said, we can make our code smaller (though somewhat less efficient) by simply sorting the strings (so that they are both the lexicographically lowest anagram) and then comparing them.

\inputcpp{code/string/is_anagram_1.cpp}

\subsubsection{Lexicographic Order}

Lexicographic order of strings is one where strings are sorted based on the alphabetical ordering of their first characters. When strings have the same first character, the alphabetical order of their second characters are considered, and so on. A similar tie-breaker has to appear with shorter strings, where the end of the string is considered first alphabetically.

For example, \{"abc", "abd", "ada", "ae", "aeiou", "b"\} is a lexicographically ordered array of strings, following the rules above.

Most languages provide some way to check whether a string is lexicographically greater or lesser than another. In C++ and Python, you can simply use \mintinline{cpp}{"abc" < "zzz"} which will return \mintinline{cpp}{true} since "abc" is lexicographically lesser than "zzz". In Java, we would want to use \mintinline{java}{"abc".compareTo("zzz")} which would return a negative value to indicate that it's lexicographically lesser (a positive value means lexicographically greater, and a value of 0 means they're the same string).

\subsubsection{Palindromes}

Palindromes are strings that are the same when they're reversed. For example, "abccba" or "abcba" are both palindromes, whereas "abcde" isn't.

It can be pretty easy to check if a string is a palindrome or not. We can simply check whether the first half of characters are equal to the second half of characters.

\inputcpp{code/string/is_palindrome.cpp}

Something to watch out for is that you can have palindromes of even length (such as "abba") and palindromes of odd length (such as "abcba"). Even length palindromes can be thought of as a string concatenated by its reverse ("ab" + "ba"), while odd length palindromes are a string concatenated by its reverse with another character in the middle ("ab" + "c" + "ba"). Often when dealing with palindromes, we may have to generate two different cases to deal with both types of palindromes.

It is also worth noting that a palindrome of length $> 2$ always contains at least one other palindrome as a substring, which we can obtain by removing the first and last character. "abba" contains "bb" as a palindrome substring, while "abcba" contains "bcb" as a palindrome substring, which itself contains "c" as a palindrome substring (of length 1). It turns out that we can divide the length of a palindrome by 2 rounded down, and get the number of palindrome substrings that are centered on the same point ("abcba"'s length divided by 2 is 2, which is the number of strict substring palindromes with "c" as the center, "bcb" and "c". The same concept applies to even length palindromes). Note that these aren't necessarily the only palindrome substrings possible -- "aabbaa" contains \{"a", "b", "aa",""bb","abba"\} as its unique palindrome substrings, as something to watch out for.

\subsubsection{Prefixes}

Prefixes are substrings that contain the start of a string. For example, the prefixes of "abcde" are \{"a","ab","abc","abcd","abcde"\}.

A strict or proper prefix is a prefix that isn't the same length as the full string (equivalently, a prefix that doesn't contain the end). Using the above example, proper prefixes would be \{"a","ab","abc","abcd"\} which notably excludes the original string "abcde".

\subsubsection{Substrings}

Substrings are the string equivalent to subarray -- it is the original string with some amount from the start and end removed. "abcd" has the substrings \{"a", "b", "c", "d", "ab", "bc", "cd", "abc", "bcd", "abcd"\}. Note that the original string "abcd" is a valid substring of itself, it just removes 0 characters from the start or the end.

Sometimes you will see a "strict" substring. This is the same as a regular substring except the original string isn't counted. This makes sure that substrings will always remove at least one character from either the start or the end, and that the length of each strict substring will be less than the original string.

\subsubsection{Suffixes}

Opposite to prefixes, suffixes are substrings that contain the end of a string. For example, the suffixes of "abcde" are \{"e","de","cde","bcde","abcde"\}.

Strict or proper suffixes are also similar to prefixes, in that they're suffixes that aren't the same length as the original string (or are suffixes that don't contain the start of the string).

\subsection{Pattern Matching}

One of the common tasks needed for string problems involves pattern matching -- determining whether or not a substring exists in the larger string altogether.

There are a fair amount of possible variants and alternatives to pattern matching. The simplest is simply "does a given substring exist in a given larger string" with no other complications. In this case, if we have some text "abcdef" and we're looking for the pattern "def", we can confidently say the pattern does exist. A common variant is to find the specific location of a matched pattern, where the previous example would start its match at position 3 in the original string. Sometimes it's sufficient to find one match even if there may be many, sometimes it's necessary to find every valid match.

\subsubsection{Regex}

\textit{Note: the term 'regex' is a common shorthand for 'regular expressions', but there is a large difference between modern regex engines and true regular expressions. Briefly, that regular expressions are regular in a formal languages sense, while regex engines include many features that make them not regular expressions. Because of this, when we say "regex" we mean the language used in modern regex engines, while "regular expressions" is more akin to theoretical computer science. Confusingly, we might also say 'regex expression' when we want to refer to a specific pattern of regex, while we don't use 'regular expression expression'.}

Pattern matching can be solved easily using regex, which is a domain-specific language that defines patterns of text. This can easily be used to search for a specific pattern, which can be as simple as a raw string such as \mintinline{python}{"pattern"}. For example, consider in python:

\inputpython{code/string/regex.py}

The main advantages of regex are that they can be very quick to code, and they can support more advanced patterns. For example, if you wanted to match either "Hello" or "World", you could use \mintinline{python}{"(Hello)|(World)"} as the pattern. You could search for any consecutive pair of vowels with \mintinline{python}{"[aeiou]{2}"}, or any word that ends in s with \mintinline{python}{"\b.+s\b"}. Regex can define fairly complex patterns to search for in only one short line. We won't go into detail here about regex's syntax or how to write complicated patterns with it, but it can be a useful read.

This comes with some hefty downsides though: namely that regex matching is significantly slower than other methods of pattern matching (to the point of needing to be very careful, because exceeding the timelimit with regex is very common), that it requires an entire different syntax to write and is generally harder to understand once written, and that if the regex engine doesn't do exactly what you need then it can't be easily modified to work.

\subsubsection{Knuth-Morris-Pratt}

Consider the naive approach: for every position $i$ in the original string $s$ with the search pattern $p$, we check if $s[i] = p[0]$, then if $s[i+1] = p[1]$, and so on until we go through the either the full pattern (in which case we found a match) or the full original string (in which case we didn't find a match).

That approach is not ideal, because we can filter out an invalid match early in many cases. For example, consider trying to match "abcabd" into "abcabcabd". When we fail to match at position 0, we've already made comparisons for positions 1 and 2 and can know that our pattern can't exist there either. This results in an unnecessary (and potentially large) slowdown.

Knuth-Morris-Pratt (commonly abbreviated as KMP) is an algorithm that can apply the necessary information to skip starting positions we may already know are invalid. We do have to do some pre-processing on the pattern string first though.

To begin, we generate an array `l` where $l[i]$ is the length of the longest proper prefix of the substring $p[0:i]$ that's also a proper suffix. In other words, for every $i$, $l[i]$ is the largest value $m$ where $m < i$ and $p[0:m] = p[i-m:i]$. The l-array of "aaa" will be $[0,1,2]$ because at $i=0$ there is no common prefix/suffix, at $i=1$ we have the substring "aa" and the longest proper prefix that's also a suffix is "a", and at $i=2$ our substring "aaa" has the longest proper prefix/suffix of "aa". Meanwhile, the l-array of "abc" is $[0,0,0]$ because it never shares a common proper prefix and suffix.

Now that we have the l-array, we can actually perform our pattern matching. Let's consider the example of trying to pattern match "abcabd" into "abcabcabd" again. We begin naively, comparing $s[0] = p[0]$ through to $s[5] = p[5]$ and notice we have a mismatch. Rather than shifting the string by 1 and recomparing the full pattern, we instead shift the string by 3 and begin comparing the pattern starting at $s[3+2] = p[2]$.

\subsubsection{Boyer Moore}
\subsubsection{Z-Algorithm}

The Z-Algorithm generates an array $Z$ where $Z[i]$ is the length of the longest common prefix between a string and its substring beginning at position $i$. Explained practically, $Z[i]$ is the number of characters that are the same starting from $s[0]$ and $s[i]$.

While this is a general string algorithm with several applications, one application is in string matching. Given text $s$ and a pattern $p$ (and some character not present in either, which we'll call $x$ which can be something like \mintinline{cpp}{'\n'} that is uncommon to be needed in a problem), we generate the z-array of $p+x+s$ so that they're all concatenated into one string.

Now, any time the length of $p$ appears inside the z-array, we know that pattern has a match. We can convince ourselves of this pretty easily -- because of $x$ we should never have a value that's greater than the length of $p$, and any value in the z-array that appears after $x$ is how many leading characters are the same in $p$.

\subsubsection{Aho Corasick}

Aho Corasick is an algorithm for pattern matching that's designed to search for multiple patterns simultaneously. Aho Corasick is based on construction of a finite state machine that represents all patterns simultaneously, and allows you to iterate over the original string with each character being used for the state transitions, and noting every accepted state as a successful match.

\subsubsection{Pattern Matching Applications}

Sometimes, we may want to do some creative applications using our pattern matching algorithms.

Consider the regex expression \mintinline{python}{"abc.*xyz"}, which essentially matches any substring that starts with "abc" and ends with "xyz", including "abcxyz". We don't actually need regex to solve this.

\subsection{Subsequence Matching}
\subsection{Hashing}
\subsection{Prefix/Suffix Arrays}
\subsection{Distance}
\subsubsection{Diff}
\subsubsection{Levenshtein}

\hrulefill

Pattern Matching:
\begin{enumerate}
\item https://open.kattis.com/problems/avion
\item https://open.kattis.com/problems/fiftyshades
\item https://open.kattis.com/problems/deathknight
\item https://open.kattis.com/problems/ostgotska
\item https://open.kattis.com/problems/geneticsearch
\item https://open.kattis.com/problems/scrollingsign
\item https://open.kattis.com/problems/stringmatching
\item https://open.kattis.com/problems/stringmultimatching
\item https://open.kattis.com/problems/messages
\end{enumerate}

Suffix Arrays:
\begin{enumerate}
\item https://open.kattis.com/problems/suffixsorting
\item https://open.kattis.com/problems/substrings
\item https://open.kattis.com/problems/dvaput
\item https://open.kattis.com/problems/burrowswheeler
\end{enumerate}
