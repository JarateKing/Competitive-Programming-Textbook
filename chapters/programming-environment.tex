\section{Programming Environment}
\index{Programming Environment}

\subsection{C++}
\index{C++}

\subsubsection{Setup}

There are a few different possible compilers that C++ can use. While options like the clang compiler are occasionally used, the most common compiler in competitive programming is GCC. Because of this, we will focus primarily on setting up the GCC compiler. The C++ compiler specifically is the \mintinline{bash}{g++} command, most often.

On a Linux system the installation varies between different distros, but is generally relatively easy for someone used to the operating system. On Ubuntu for example, you should be able to run \mintinline{bash}{sudo apt install build-essential} which will install various necessary packages, including \mintinline{bash}{g++}.

On Mac OSX...

On windows, the setup requires \url{http://mingw.org/} in order to get the GCC. You can download the installation manager that will simplify the process.

To verify that the install worked, you should be able to run \mintinline{bash}{g++ --version} to see if it's a missing command or if it tells you the version number.

\subsubsection{Strengths and Weaknesses}

The most clear benefit of C++ is that it's among the fastest languages there are, and is usually the fastest language available in a contest setting. While solutions in other languages will usually be able to pass a problem's time constraints (assuming the algorithm is the intended solution, and potentially with some optimization) C++ is the only language that is guaranteed to be capable of passing time constraints.

While not an actual feature of the language itself, C++ is also the most common language in competitive programming in general. As such, you will find most resources use C++.

Unfortunately, there are some notable issues in C++'s standard library. For example, the standard library offers no way to represent arbitrarily large integers, and you would have to implement a class for this purpose on your own.

\subsection{Java}
\index{Java}

\subsubsection{Setup}

Under Construction

\subsubsection{Strengths and Weaknesses}

Compared to the other languages listed here, Java's strength is that it enforces much more safeguards at compile-time than either C++ (that often only warns about things if you enable those warnings in the compiler, and they are not necessarily the most clear warnings) or python (that doesn't even have a proper compile step, and outside or erroneous syntax will throw its errors at runtime).

Java also comes with some very useful bits inside its standard library. An example would be \mintinline{java}{BigInteger} and \mintinline{java}{BigDecimal} that...

Unfortunately, it does come with a fair amount of boilerplate, so solutions in Java tend to be relatively verbose compared to other languages.

While Java's JIT is very impressive and its programs are quite fast once it's warmed up, this startup time is not an insignificant concern.

\subsection{Python}
\index{Python}

For the purposes of this book, we assume Python 3 is used. Python 2 can offer better speeds on some platforms due to more optimized interpreters and compilers, but in general we aren't too concerned about that as far as this book is concerned. If you would prefer to use Python 2, most of the code examples given should be fairly simple to translate.

\subsubsection{Setup}

Under Construction

\subsubsection{Strengths and Weaknesses}

Python's main strength from a competitive programming standpoint is that it has very little boilerplate code, and in general it is very short. Many of its abstractions can lead to very short code, and in fact lots of relatively simple (at least, simple to code) problems can be solved in a single line.

The main weakness of Python is that it's fairly slow, and is certainly slower than C++ or Java. Some contests will specifically point out that they cannot guarantee Python is able to pass the time constraints (though this only means the contest testers didn't write a solution in Python, not that they actually know it can't be done).

\subsection{Other}

While we won't go in-depth here describing them, there are a few other languages that are important to mention. You can explore them but this book isn't going to describe setting them up or cover anything about them in any sort of detail, in favor of the other three languages C++, Java, and Python.

\subsubsection{C}
\index{C}

C has some historical note as one of the languages offered in many competitions.

In practice, C++ is generally preferred -- there isn't any major speed differences between the two, and for the most part C++ is a superset of C with many more useful utilities and parts to its standard library. While there is valid reasoning why you would prefer C for some general software development (usually relating to what platforms it can run on or the team that you develop with), none applies for competitive programming.

\subsubsection{Pascal}
\index{Pascal}

Like C, Pascal has some historical note as well. It was a fairly popular language in IOI competitions and was one of the few allowed, though in recent years it was removed from the list of valid languages.

\subsubsection{Kotlin}
\index{Kotlin}

Kotlin has started to see some use and is being offered in some contests recently.

\subsection{Text Editors}
\index{Text Editors}

Most text editors are comparable to each other in terms of quality and features, and usage is largely based on preference. That said, there are some recommendations to be made.

In Windows, Notepad++ is a common recommendation. Gedit and Geany are often options available on Linux systems (and have the benefit that they are usually available on contest computers). Sublime, Atom, Visual Studio Code, or many others are viable options as well on multiple platforms.

IDEs (integrated development environments) are very nice and convenient, but if you intend on entering competitions like the ICPC where IDEs are usually not available, it is recommended to avoid (or at least be comfortable without) using an IDE. They are also often relatively bulky, with large menus and slowdowns due to features that are only really relevant in larger-scale projects.

Usually, text editors like vim and emacs are offered in contest settings. However, it's important to consider that you would not have much time to set up your editor to your preferences, and a complicated setup would be hard to replicate or take some time to prepare. For that reason, if you do decide to use an editor like vim or emacs, it's worthwhile to practice with them either being completely default, or with very little customization that you can quickly apply in a contest.