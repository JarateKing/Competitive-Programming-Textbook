\section{Geometry Problems}
\index{Geometry}

\subsection{Primitives}
\index{Primitives}

There are various simple constructs in geometry that we have to use frequently. These are called "primitives" here, and can be thought of as the basic building blocks of any geometry problem.

\subsubsection{Points}
\index{Points}

Points are the most basic construct in geometry -- they simply indicate a specific location.

In terms of code, a simple point structure can be created by simply storing a variable for each dimension. For example, with a 2D point:

\inputcpp{code/geometry/point.cpp}

\subsubsection{Lines}
\index{Lines}

Lines are a logical extension of points -- they represent a point and a direction.

There are a variety of ways to define a line. One is through two points. Another is through a point and an angle, which determines the direction.

\subsubsection{Segments}
\index{Segments}

Segments are akin to lines, except they have a fixed distance instead of an infinite length.

Like lines, there are more than one way to define segments. Two points is another option, except in segments they are the start and end points rather than two arbitrary points along the line. A point, an angle, and a magnitude can be used to define a segment as well.

\subsubsection{Circles}
\index{Circles}

\subsubsection{Triangles}
\index{Triangles}

\subsubsection{Rectangles}
\index{Rectangles}

\subsubsection{Polygons}
\index{Polygons}

\subsubsection{Convex Polygons}
\index{Convex Polygons}

\hrulefill

Basic:
\begin{enumerate}
\item https://open.kattis.com/problems/beavergnaw
\item https://open.kattis.com/problems/movingday
\item https://open.kattis.com/problems/sanic
\end{enumerate}
