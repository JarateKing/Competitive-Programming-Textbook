\section{Geometry Problems}
\index{Geometry}

\subsection{Primitives}
\index{Primitives}

There are various simple constructs in geometry that we have to use frequently. These are called "primitives" here, and can be thought of as the basic building blocks of any geometry problem.

\subsubsection{Points}
\index{Points}

Points are the most basic construct in geometry -- they simply indicate a specific location.

In terms of code, a simple point structure can be created by simply storing a variable for each dimension. For example, with a 2D point:

\inputcpp{code/geometry/point.cpp}

\subsubsection{Lines}
\index{Lines}

Lines are a logical extension of points -- they represent a point and a direction.

There are a variety of ways to define a line. One is through two points. Another is through a point and an angle, which determines the direction.

\subsubsection{Segments}
\index{Segments}

Segments are akin to lines, except they have a fixed distance instead of an infinite length.

Like lines, there are more than one way to define segments. Two points is another option, except in segments they are the start and end points rather than two arbitrary points along the line. A point, an angle, and a magnitude can be used to define a segment as well.

\subsubsection{Circles}
\index{Circles}

Circles are a slightly more complicated structure than the previous ones explored so far. They can be defined simply as a point and a radius, where the circle itself is all the (infinite) points that are exactly the radius away from the point (which acts as the center of the circle).

While this definition itself may not be overly complicated, we have various other things to consider with this structure, and various circle-specific terms to define. The diameter is the width of the circle, or the distance between one point on a circle and the opposite point -- this is easily calculated as twice the radius. The circumference, which in plain terms is the length of the circle if it was unrolled into a straight segment, is calculated as $2 \pi r$. The area of a circle is defined as $\pi r^2$.

We can also define a circle by three points. We won't cover how to derive this, and that is left as an exercise for anyone interested. The code is included:

\inputcpp{code/geometry/circle.cpp}

Note that these aren't the only ways to define a circle. The equation $(x-h)^2 + (y-k)^2 = r^2$ where $\{h,k\}$ is point is commonly used to define a circle and can express it on the coordinate plane. This definition is useful to keep in mind for some problems, but generally very easy to convert into a point and a radius, so it is not included here.

\subsubsection{Triangles}
\index{Triangles}

\subsubsection{Rectangles}
\index{Rectangles}

\subsubsection{Polygons}
\index{Polygons}

\subsubsection{Convex Polygons}
\index{Convex Polygons}

\hrulefill

Basic:
\begin{enumerate}
\item https://open.kattis.com/problems/beavergnaw
\item https://open.kattis.com/problems/movingday
\item https://open.kattis.com/problems/sanic
\end{enumerate}
